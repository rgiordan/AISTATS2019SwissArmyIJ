
%%%%%%%%%%%%%%%%%%%%%%%
% BEGIN LYX STUFF

% Please don't edit this section manually. It's fussy and hard to get it to work.
% As it currently is, we can incorporate files from LyX.

% Please copy these from LyX so we can use both styles of writing
\AtBeginDocument{\providecommand\assuref[1]{\ref{assu:#1}}}
\AtBeginDocument{\providecommand\condref[1]{\ref{cond:#1}}}
\AtBeginDocument{\providecommand\lemref[1]{\ref{lem:#1}}}
\AtBeginDocument{\providecommand\defrefref[1]{\ref{defref:#1}}}
\AtBeginDocument{\providecommand\proprefref[1]{\ref{propref:#1}}}
\AtBeginDocument{\providecommand\thmrefref[1]{\ref{thmref:#1}}}
\AtBeginDocument{\providecommand\corref[1]{\ref{corref:#1}}}

\theoremstyle{plain}
    \ifx\thechapter\undefined
      \newtheorem{assumption}{\protect\assumptionname}
    \else
      \newtheorem{assumption}{\protect\assumptionname}[chapter]
    \fi
\theoremstyle{definition}
    \ifx\thechapter\undefined
      \newtheorem{defn}{\protect\definitionname}
    \else
      \newtheorem{defn}{\protect\definitionname}[chapter]
    \fi
\theoremstyle{plain}
    \ifx\thechapter\undefined
      \newtheorem{condition}{\protect\conditionname}
    \else
      \newtheorem{condition}{\protect\conditionname}[chapter]
    \fi
\theoremstyle{plain}
    \ifx\thechapter\undefined
      \newtheorem{prop}{\protect\propositionname}
    \else
      \newtheorem{prop}{\protect\propositionname}[chapter]
    \fi
\theoremstyle{plain}
    \ifx\thechapter\undefined
      \newtheorem{lem}{\protect\lemmaname}
    \else
      \newtheorem{lem}{\protect\lemmaname}[chapter]
    \fi
\theoremstyle{plain}
    \ifx\thechapter\undefined
	    \newtheorem{thm}{\protect\theoremname}
	  \else
      \newtheorem{thm}{\protect\theoremname}[chapter]
    \fi
\theoremstyle{plain}
    \ifx\thechapter\undefined
  \newtheorem{cor}{\protect\corollaryname}
\else
      \newtheorem{cor}{\protect\corollaryname}[chapter]
    \fi

%%%%%%%%%%%%%%%%%%%%%%%%%%%%%% User specified LaTeX commands.
\newref{cond}{refcmd={Condition \ref{#1}}}
\newref{assu}{refcmd={Assumption \ref{#1}}}
\newref{propref}{refcmd={Proposition \ref{#1}}}
\newref{corref}{refcmd={Corrolary \ref{#1}}}
\newref{lem}{refcmd={Lemma \ref{#1}}}
\newref{defref}{refcmd={Definition \ref{#1}}}
\newref{thmref}{refcmd={Theorem \ref{#1}}}

\makeatother

%\usepackage{babel}
\providecommand{\assumptionname}{Assumption}
\providecommand{\conditionname}{Condition}
\providecommand{\corollaryname}{Corollary}
\providecommand{\definitionname}{Definition}
\providecommand{\lemmaname}{Lemma}
\providecommand{\propositionname}{Proposition}
\providecommand{\theoremname}{Theorem}

\global\long\def\at#1{\rvert_{#1}}

\global\long\def\norm#1{\left\Vert #1\right\Vert }

\global\long\def\mbe{\mathbb{E}}

\global\long\def\thetahat{\hat{\theta}}

\global\long\def\onevec{1_{w}}

\global\long\def\thetawiggle{\tilde{\theta}}

\global\long\def\thetaone{\hat{\theta}_{1}}

\global\long\def\thetaij{\hat{\theta}_{\textrm{IJ}}}

\newcommandx\thetaw[1][usedefault, addprefix=\global, 1=w]{\hat{\theta}\left(#1\right)}

\global\long\def\hone{H_{1}}

\global\long\def\constij{C_{\textrm{IJ}}}

\global\long\def\consth{C_{h}}

\global\long\def\constg{C_{g}}

\global\long\def\constw{C_{w}}

\newcommandx\constop[1][usedefault, addprefix=\global, 1={\text{\,}}]{C_{op}^{#1}}

\global\long\def\liph{L_{h}}

\global\long\def\thetasize{\Delta_{\theta}}

\global\long\def\deltasize{\Delta_{\delta}}

\global\long\def\thetaball{B_{\Delta_{\theta}}}

\global\long\def\thetadeltaball{B_{C_{op}\delta}}

\global\long\def\hint{\tilde{H}}


% END LYX STUFF
%%%%%%%%%%%%%%%%%%%%%%%
