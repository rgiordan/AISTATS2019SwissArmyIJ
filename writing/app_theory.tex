\newcommand{\globalassum}{Assumptions \ref{assu:paper_smoothness}---\ref{assu:paper_bounded} }

\section{Detailed assumptions, lemmas, and proofs\label{sec:appendix_proofs}}





\subsection{Tools}

We begin by stating two general propositions that will be useful.
First, we show that a version of Cauchy-Schwartz can be applied to
weighted sums of tensors.
\begin{prop}
\label{propref:tensor_cauchy_schwartz}Tensor array version of H{\"o}lder's
inequality. Let $w$ be an array of scalars and let $a=\left(a_{1},...,a_{N}\right)$
be an array of tensors, were each $a_{n}$ is indexed by $i=1,\ldots,D_{A}$
($i$ may be a multi-index---e.g., if $A$ is a $D\times D$ matrix,
then $i=\left(j,k\right)$, for $j,k\in\left[D\right]$ and $D_{A}=D^{2}$).
Let $p,q\in\left[1,\infty\right]$ be two numbers such that $p^{-1}+q^{-1}=1$.
Then
\begin{align*}
\norm{\frac{1}{N}\sum_{n=1}^{N}w_{n}a_{n}}_{1} & \le\frac{D_{A}^{\frac{1}{p}}}{N}\norm w_{p}\norm a_{q}.
\end{align*}
In particular, with $p=q=2$,
\begin{align*}
\norm{\frac{1}{N}\sum_{n=1}^{N}w_{n}a_{n}}_{1} & \le\sqrt{D_{A}}\frac{\norm w_{2}}{\sqrt{N}}\frac{\norm a_{2}}{\sqrt{N}}.
\end{align*}
\end{prop}
\begin{proof}
The conclusion follows from the ordinary H{\"o}lder's inequality
applied term-wise to $n$ and Jensen's inequality applied to the indices
$i$.

\begin{align*}
\norm{\frac{1}{N}\sum_{n=1}^{N}w_{n}a_{n}}_{1} & =\sum_{i=1}^{D_{A}}\left|\frac{1}{N}\sum_{n=1}^{N}w_{n}\left(a_{n}\right)_{i}\right|\\
 & \le\frac{1}{N}\sum_{i=1}^{D_{A}}\left|\left(\sum_{n=1}^{N}\left|w_{n}\right|^{p}\right)^{\frac{1}{p}}\left(\sum_{n=1}^{N}\left|\left(a_{n}\right)_{i}\right|^{q}\right)^{\frac{1}{q}}\right|\text{(H{\"o}lder)}\\
 & =\frac{1}{N}\norm w_{p}\frac{D_{A}}{D_{A}}\sum_{i=1}^{D_{A}}\left(\sum_{n=1}^{N}\left|\left(a_{n}\right)_{i}\right|^{q}\right)^{\frac{1}{q}}\\
 & \le\frac{1}{N}\norm w_{p}D_{A}\left(\frac{1}{D_{A}}\sum_{i=1}^{D_{A}}\sum_{n=1}^{N}\left|\left(a_{n}\right)_{i}\right|^{q}\right)^{\frac{1}{q}}\textrm{ (Jensen applied to }i\textrm{)}\\
 & =\frac{1}{N}\norm w_{p}D_{A}\left(\frac{1}{D_{A}}\sum_{n=1}^{N}\norm{a_{n}}_{q}^{q}\right)^{\frac{1}{q}}\\
 & =\frac{1}{N}\norm w_{p}D_{A}^{1-\frac{1}{q}}\norm a_{q}\\
 & =\frac{D_{A}^{\frac{1}{p}}}{N}\norm w_{p}\norm a_{q}.
\end{align*}
\end{proof}
%
Next, we prove a relationship between the term-wise difference between
matrices and the difference between their operator norms. It is well-known
that the minimum eigenvalue of a non-singular matrix is continuous
in the entries of the matrix. In the next proposition, we quantify
this continuity for the $L_{1}$ norm.
\begin{prop}
\label{propref:operator_norm_continuity}Let $A$ and $B$ be two
matrices. Let $\norm{A^{-1}}_{op}\le\constop$ for some finite $\constop$,
Then
\begin{align*}
\norm{A-B}_{1}\le\frac{1}{2}\constop[-1] & \quad\Rightarrow\quad\norm{B^{-1}}_{op}\le2\constop.
\end{align*}
\end{prop}
\begin{proof}
We will use \citet[Theorem 5.20]{schott:2016:matrix} and the associated
discussion, which states the following general result. Take any matrix
norm $\norm{\cdot}$ that satisfies $\norm I=1$, where $I$ is the
identity matrix. Then if $\norm{A^{-1}}\norm{A-B}<1$, then
\begin{align}
\norm{A^{-1}-\left(A-B\right)^{-1}} & \le\frac{\norm{A^{-1}}\norm{A-B}}{1-\norm{A^{-1}}\norm{A-B}}\norm{A^{-1}}.\label{eq:matrix_norm_continuity}
\end{align}

We will apply \eqref{matrix_norm_continuity} using the operator norm
$\norm{\cdot}_{op}$, for which $\norm I_{op}=1$. First, note that
\begin{align*}
\norm{A^{-1}}_{op}\norm{A-B}_{op} & \le\norm{A^{-1}}_{op}\norm{A-B}_{1}\textrm{ (ordering of matrix norms)}\\
 & \le\frac{1}{2}\constop\constop[-1]\\
 & =\frac{1}{2},
\end{align*}

so we can apply \eqref{matrix_norm_continuity}. Then
\begin{align*}
\norm{B^{-1}}_{op}
 & \le \norm{B^{-1}-A^{-1}}_{op}+\norm{A^{-1}}_{op}\textrm{ (triangle inequality)}\\
 & \le\frac{\norm{A^{-1}}_{op}\norm{A-B}_{op}}{1-\norm{A^{-1}}_{op}\norm{A-B}_{op}}\norm{A^{-1}}_{op}+\norm{A^{-1}}_{op}\textrm{ (Equation \ref{eq:matrix_norm_continuity})}\\
 & \le\left(\frac{\frac{1}{2}}{1-\frac{1}{2}}+1\right)\norm{A^{-1}}\\
 & \le2\constop.
\end{align*}
\end{proof}








\subsection{Lemmas}

We now prove some useful consequences of our assumptions. The proof
roughly proceeds for all $w\in W_{\delta}$ by the following steps:
\begin{enumerate}
\item When $\delta$ is small we can make $\norm{{{\thetaw}}-\thetaone}_{2}$
small. (\lemref{theta_difference_bound} below.)
\item When $\norm{\theta-\thetaone}_{2}$ is small, then the derivatives
$H\left(\theta,w\right)$ are close to their optimal value, $H\left(\thetaone,\onevec\right)$.
(\lemref{bounded_continuous} and \lemref{gh_difference_from_one}
below.)
\item When the derivatives are close to their optimal values, then $H\left(\theta,w\right)$
is uniformly non-singular. (\lemref{continuous_invertibility} below.)
\item When the derivatives are close to their optimal values and $H\left(\theta,w\right)$
is uniformly non-singular we can control the error in $\thetaij-\thetaw$
in terms of $\delta$. (\thmrefref{taylor_error_first} below.)
\end{enumerate}
We begin by showing that the difference between $\thetaw$ and $\thetaone$
for $w\in W_{\delta}$ can be made small by making $\delta$ from
\condref{paper_uniform_bound} small.

\begin{lem}
\label{lem:theta_difference_bound}Small parameter changes. Under
\globalassum and \condref{paper_uniform_bound},
\begin{align*}
\textrm{for all }w\in W_{\delta},\quad\norm{{{\thetaw}}-\thetaone}_{2} & \le\constop\delta.
\end{align*}
\end{lem}
\begin{proof}
By a first-order Taylor expansion in $\theta$, for some $\tilde{\theta}$
such that $\norm{\tilde{\theta}-\thetaone}_{2}\le\norm{{{\thetaw}}-\thetaone}_{2}$,
\begin{align*}
G\left(\thetaw,\onevec\right) & =G\left(\thetaone,\onevec\right)+H\left(\tilde{\theta},\onevec\right)\left(\thetaw-\thetaone\right).
\end{align*}
By \assuref{paper_hessian}, $H\left(\tilde{\theta},\onevec\right)$
is non-singular. A little manipulation, together with the fact that
$G\left(\thetaw,w\right)=G\left(\thetaone,\onevec\right)=0$ gives
\begin{align*}
G\left(\thetaw,\onevec\right)-G\left(\thetaw,w\right) & =H\left(\tilde{\theta},\onevec\right)\left(\thetaw-\thetaone\right)\Rightarrow\\
\thetaw-\thetaone & =H\left(\tilde{\theta},\onevec\right)^{-1}\left(G\left(\thetaw,\onevec\right)-G\left(\thetaw,w\right)\right).
\end{align*}
Applying \condref{paper_uniform_bound} and \assuref{paper_hessian},
\begin{align*}
\norm{{{\thetaw}}-\thetaone}_{2} & =\norm{H\left(\tilde{\theta},\onevec\right)^{-1}\left(G\left({{\thetaw}},\onevec\right)-G\left({{\thetaw}},w\right)\right)}_{2}\\
 & \le\norm{H\left(\tilde{\theta},\onevec\right)^{-1}}_{op}\norm{\left(G\left({{\thetaw}},\onevec\right)-G\left({{\thetaw}},w\right)\right)}_{2}\\
 & \le\sup_{\theta\in\Omega_{\theta}}\norm{H\left(\theta,\onevec\right)^{-1}}_{op}\norm{\left(G\left({{\thetaw}},\onevec\right)-G\left({{\thetaw}},w\right)\right)}_{2}\\
 & \le\constop\norm{G\left({{\thetaw}},\onevec\right)-G\left({{\thetaw}},w\right)}_{2}\textrm{ (Assumption \ref{assu:paper_hessian})}\\
 & \le\constop\norm{G\left({{\thetaw}},\onevec\right)-G\left({{\thetaw}},w\right)}_{1}\textrm{ (relation between norms)}\\
 & \le\constop\sup_{\theta\in\Omega_{\theta}}\norm{G\left(\theta,\onevec\right)-G\left(\theta,w\right)}_{1}\\
 & \le\constop\delta.\textrm{ (Condition \ref{cond:paper_uniform_bound}).}
\end{align*}
\end{proof}
%
Because we will refer to it repeatedly, we give the set of $\theta$
defined in \lemref{theta_difference_bound} a name.
\begin{defn}
For a given $\delta$, define the region around $\thetaone$ given
by \lemref{theta_difference_bound} as

\begin{align*}
\thetadeltaball & :=\left\{ \theta:\norm{\theta-\thetaone}_{2}\le\constop\delta\right\} \bigcap\Omega_{\theta}.
\end{align*}
\end{defn}
%
In other words, \lemref{theta_difference_bound} states that \condref{paper_uniform_bound}
implies $\thetaw\in\thetadeltaball$ when $w\in W_{\delta}$.

Next, we show that closeness in $\theta$ will mean closeness in $H\left(\theta,w\right)$.
\begin{lem}
\label{lem:bounded_continuous} Boundedness and continuity. Under
\paperallcoreassum and \condref{paper_uniform_bound},
\begin{align*}
\textrm{for all }\theta\in\thetaball,\quad\sup_{w\in W}\norm{H\left(\theta,w\right)-H\left(\thetaone,w\right)}_{1} & \le D\constw\liph\norm{\theta-\thetaone}_{2}.
\end{align*}
\end{lem}
\begin{proof}
For $\theta\in\thetaball$,
\begin{align*}
\sup_{w\in W}\norm{H\left(\theta,w\right)-H\left(\thetaone,w\right)}_{1} & =\sup_{w\in W}\norm{\frac{1}{N}\sum_{n=1}^{N}w_{n}\left(h_{n}\left(\theta\right)-h_{n}\left(\thetaone\right)\right)}_{1}\textrm{ (by definition)}\\
 & \le D\sup_{w\in W}\frac{\norm w_{2}}{\sqrt{N}}\frac{\norm{h\left(\theta\right)-h\left(\thetaone\right)}_{2}}{\sqrt{N}}\textrm{ (Proposition \ref{propref:tensor_cauchy_schwartz})}\\
 & \le D\constw\frac{\norm{h\left(\theta\right)-h\left(\thetaone\right)}_{2}}{\sqrt{N}}\textrm{ (Assumption \ref{assu:paper_weight_bounded})}\\
 & \le D\constw\liph\norm{\theta-\thetaone}_{2}\textrm{ (Assumption \ref{assu:paper_lipschitz} and }\theta\in\thetaball\textrm{)}.
\end{align*}
\end{proof}
%
We now combine \lemref{theta_difference_bound} and \lemref{bounded_continuous}
to show that $H\left(\theta,w\right)$ is close to its value at the
solution $H\left(\thetaone,\onevec\right)$ for sufficiently small
$\delta$ and for all $\theta\in\thetadeltaball$.
%
\begin{lem}
\label{lem:gh_difference_from_one}Bounds for difference in parameters.
Under \paperallcoreassum and \condref{paper_uniform_bound}, if $\delta\le\thetasize\constop[-1]$,
then
\begin{align*}
\sup_{\theta\in\thetadeltaball}\sup_{w\in W_{\delta}} & \norm{H\left(\theta,w\right)-H\left(\thetaone,\onevec\right)}_{1}\le\left(1+D\constw\liph\constop\right)\delta.
\end{align*}
\end{lem}
\begin{proof}
By \lemref{theta_difference_bound}, $\delta\le\thetasize\constop[-1]$
implies that $\constop\delta\le\thetasize$ and so $\thetadeltaball\subseteq\thetaball$.
Consequently, we can apply \lemref{bounded_continuous}:
\begin{align*}
\sup_{\theta\in\thetadeltaball}\sup_{w\in W_{\delta}}\norm{H\left(\theta,w\right)-H\left(\thetaone,w\right)}_{1} & \le\sup_{\theta\in\thetaball}\sup_{w\in W_{\delta}}\norm{H\left(\theta,w\right)-H\left(\thetaone,w\right)}_{1}\\
 & \le D\constw\liph\norm{\theta-\thetaone}_{2}\textrm{ (Lemma \ref{lem:bounded_continuous})}\\
 & \le D\constw\liph\constop\delta\quad\textrm{ (because }\theta\in\thetadeltaball\textrm{)}.
\end{align*}
%
Next, we can use this to write
\begin{align*}
\sup_{\theta\in\thetadeltaball}\sup_{w\in W_{\delta}} & \norm{H\left(\theta,w\right)-H\left(\thetaone,\onevec\right)}_{1}\\
 & =\sup_{\theta\in\thetadeltaball}\sup_{w\in W_{\delta}}\norm{H\left(\theta,w\right)-H\left(\theta,\onevec\right)+H\left(\theta,\onevec\right)-H\left(\thetaone,\onevec\right)}_{1}\\
 & \le\sup_{\theta\in\thetadeltaball}\sup_{w\in W_{\delta}}\norm{H\left(\theta,w\right)-H\left(\theta,\onevec\right)}_{1}+\sup_{\theta\in\thetadeltaball}\sup_{w\in W_{\delta}}\norm{H\left(\theta,\onevec\right)-H\left(\thetaone,\onevec\right)}_{1}\\
 & \le\sup_{\theta\in\Omega_{\theta}}\sup_{w\in W_{\delta}}\norm{H\left(\theta,w\right)-H\left(\theta,\onevec\right)}_{1}+\sup_{\theta\in\thetadeltaball}\sup_{w\in W_{\delta}}\norm{H\left(\theta,\onevec\right)-H\left(\thetaone,\onevec\right)}_{1}\\
 & \le\delta+\sup_{\theta\in\thetadeltaball}\sup_{w\in W_{\delta}}\norm{H\left(\theta,\onevec\right)-H\left(\thetaone,\onevec\right)}_{1}\textrm{ (Condition \ref{cond:paper_uniform_bound})}\\
 & \le\delta+D\constw\liph\constop\delta.
\end{align*}
\end{proof}
%
The constant that appears multiplying $\delta$ at the end of the proof of
\lemref{gh_difference_from_one} will appear often in what follows, so we give it
the special name $\constij$ in \defrefref{constants_definition}.

% \begin{defn}
% Define the constant that appears in the Hessian bound of \lemref{gh_difference_from_one}
% as
% \begin{align*}
% \constij & :=1+D\constw\liph\constop.
% \end{align*}
% \end{defn}
%
Note that \lemref{gh_difference_from_one} places a condition on how
small $\delta$ must be in order for our regularity conditions to
apply. \lemref{theta_difference_bound} will guarantee that $\thetaw\in\thetadeltaball$,
but if we are not able to make $\delta$ arbitrarily small in \condref{paper_uniform_bound},
then we are not guaranteed to ensure that $\thetadeltaball\subseteq\thetaball$,
will not be able to assume Lipschitz continuity, and none of our results
will apply.

Next, using \lemref{gh_difference_from_one}, we can extend the operator
bound on $\hone^{-1}$ from \assuref{paper_hessian} to $H\left(\theta,w\right)^{-1}$
for all $w\in W_{\delta}$, not only for $w=\onevec$.

\begin{lem}
\label{lem:continuous_invertibility}Uniform invertibility of the
Hessian. Under \paperallcoreassum and \condref{paper_uniform_bound}, if $\delta\le\min\left\{ \thetasize\constop[-1],\frac{1}{2}\constij^{-1}\constop[-1]\right\} $,
then
\begin{align*}
\sup_{\theta\in\thetadeltaball}\sup_{w\in W_{\delta}}\norm{H\left(\theta,w\right)^{-1}}_{op} & \le2\constop.
\end{align*}
\end{lem}
\begin{proof}
By \assuref{paper_hessian}, $\norm{H\left(\thetaone,\onevec\right)^{-1}}_{op}\le\constop$.
So by \proprefref{operator_norm_continuity}, it will suffice to select
$\delta$ so that
\begin{align}
\sup_{\theta\in\thetadeltaball}\sup_{w\in W_{\delta}}\norm{H\left(\theta,w\right)-H\left(\thetaone,\onevec\right)}_{1} & \le\frac{1}{2}\constop[-1].\label{eq:h_bound}
\end{align}
When we can apply \lemref{gh_difference_from_one}, we have
\begin{align*}
\sup_{\theta\in\thetadeltaball}\sup_{w\in W_{\delta}}\norm{H\left(\theta,w\right)-H\left(\thetaone,\onevec\right)}_{1} & \le\constij\delta.
\end{align*}
So $H\left(\theta,w\right)$ will satisfy \eqref{h_bound} if we can
apply \lemref{gh_difference_from_one} and if
\begin{align*}
\delta\le & \frac{1}{2}\constop[-1]\constij^{-1}.
\end{align*}
To apply \lemref{gh_difference_from_one} we additionally require
that $\delta\le\thetasize\constop[-1]$. By taking $\delta\le\min\left\{ \thetasize\constop[-1],\frac{1}{2}\constop[-1]\constij^{-1}\right\} $,
we satisfy \eqref{h_bound} and the result follows.
\end{proof}
%
At last, the upper bound on $\delta$ will be sufficient to control
the error terms in our approximation. For compactness, we give it
the name $\deltasize$ in \defrefref{constants_definition}.
% \begin{defn}
% Define $\deltasize$ as the upper bound on $\delta$ given by \lemref{continuous_invertibility}.
% \begin{align*}
% \deltasize & :=\min\left\{ \thetasize\constop[-1],\frac{1}{2}\constop[-1]\constij^{-1}\right\} .
% \end{align*}
%
%\end{defn}

Finally, we state a result that will allow us to define derivatives
of $\thetaw$ with respect to $w$.
\begin{lem}
\label{lem:implicit_function_theorem}Inverse function theorem. Under
\paperallcoreassum and \condref{paper_uniform_bound}, and for $\delta\le\deltasize$,
there exists a continuous, differentiable function of $w$, $\thetahat\left(w\right)$,
such that, for all $w\in W$, G$\left(\thetahat\left(w\right),w\right)=0$.
\end{lem}
\begin{proof}
This follows from \lemref{continuous_invertibility} and the implicit
function theorem.
\end{proof}
By definition, $\thetahat\left(\onevec\right)=\thetaone$.






\subsection{Bounding the errors in a Taylor expansion}

We are now in a position to use \paperallcoreassum and \condref{paper_uniform_bound}
to bound the error terms in a first-order Taylor expansion of $\thetaw$.
We begin by simply calculating the derivative $d\thetahat\left(w\right)/dw$.
\begin{prop}
\label{propref:theta_w_first_derivative}For any $w\in W$ for which
$H\left(\thetaw,w\right)$ is invertible, and for any vector $a\in\mathbb{R}^{N}$,
\begin{align*}
\frac{d\thetaw}{dw^{T}}\at wa & =-H\left(\thetaw,w\right)^{-1}G\left(\thetaw,a\right).
\end{align*}
\end{prop}
\begin{proof}
Because $G\left(\thetaw,w\right)=0$ for all $w\in W$, by direct
calculation,
\begin{align*}
0 & =\frac{d}{dw^{T}}G\left(\thetaw,w\right)\at wa\\
 & =\left(\frac{\partial G}{\partial\theta^{T}}\frac{d\hat{\theta}}{dw^{T}}+\frac{\partial G}{\partial w^{T}}\right)\at{_{w}}a\\
 & =H\left(\thetaw,w\right)\frac{d\hat{\theta}}{dw^{T}}\at{_{w}}a+\left(\frac{\partial}{\partial w^{T}}\frac{1}{N}\sum_{n=1}^{N}w_{n}g_{n}\left(\theta\right)\right)\at{_{w}}a\\
 & =H\left(\thetaw,w\right)\frac{d\hat{\theta}}{dw^{T}}\at{_{w}}a+\frac{1}{N}\sum_{n=1}^{N}g_{n}\left(\thetaw\right)a\\
 & =H\left(\thetaw,w\right)\frac{d\hat{\theta}}{dw^{T}}\at{_{w}}a+G\left(\thetaw,a\right).
\end{align*}
Because $H\left(\thetaw,w\right)$ is invertible by assumption, the
result follows.
\end{proof}
\begin{defn}
\label{defref:theta_infinitesimal_jackknife}Define
\begin{align*}
\thetaij\left(w\right) & :=\thetaone+\frac{d\thetaw}{dw^{T}}\at{\onevec}\left(w-\onevec\right)\\
 & =\thetaone-\hone^{-1}G\left(\thetaone,w\right).\textrm{ (because }G\left(\thetaone,\onevec\right)=0\textrm{)}
\end{align*}
\end{defn}
%
$\thetaij\left(w\right)$ in \defrefref{theta_infinitesimal_jackknife}
is the first term in a Taylor series expansion of $\thetaw$ as a
function of $w$. We want to bound the error, $\thetaij\left(w\right)-\thetaw$.
\begin{thm}
\label{thmref:taylor_error_first}Under \paperallcoreassum and \condref{paper_uniform_bound},
when $\delta\le\deltasize$,
\begin{align*}
\sup_{w\in W_{\delta}}\norm{\thetaij\left(w\right)-\thetahat\left(w\right)}_{2} & \le2\constop[2]\constij\delta^{2}.
\end{align*}
\end{thm}
\begin{proof}
By a one-term Taylor series expansion of $G\left(\thetaw,w\right)=0$
in $\theta$ around $\thetaone$, we have, for some $\thetawiggle$
such that $\norm{\thetawiggle-\thetaone}_{2}\le\norm{{{\thetaw}}-\thetaone}_{2}$,
\begin{align*}
0=G\left(\thetaw,w\right) & =G\left(\thetaone,w\right)+H\left(\thetawiggle,w\right)\left(\thetaw[w]-\thetaone\right).
\end{align*}
Because $\delta\in W_{\delta}$, \lemref{theta_difference_bound}
implies that $\thetaw\in\thetadeltaball$. Because $\norm{\thetawiggle-\thetaone}_{2}\le\norm{{{\thetaw}}-\thetaone}_{2}$,
$\thetawiggle\in\thetadeltaball$ as well. Because $\thetawiggle\in\thetadeltaball$,
\lemref{continuous_invertibility} implies that $H\left(\thetawiggle,w\right)$
is invertible, so we can solve for $\thetaw-\thetaone$.
\begin{align*}
\thetaw[w]-\thetaone & =-H\left(\thetawiggle,w\right)^{-1}G\left(\thetaone,w\right)\\
 & =\left(-H\left(\thetawiggle,w\right)^{-1}+H\left(\thetaone,\onevec\right)^{-1}-H\left(\thetaone,\onevec\right)^{-1}\right)G\left(\thetaone,w\right)\\
 & =\left(H\left(\thetaone,\onevec\right)^{-1}-H\left(\thetawiggle,w\right)^{-1}\right)G\left(\thetaone,w\right)+\thetaij\left(w\right)-\thetaone.
\end{align*}
Eliminating $\thetaone$ and taking the supremum of both sides we
have that
\begin{align*}
\sup_{w\in W_{\delta}} & \norm{\thetaij\left(w\right)-{{\thetaw[w]}}}_{2}\\
 & =\sup_{w\in W_{\delta}}\norm{\left(H\left(\thetaone,\onevec\right)^{-1}-H\left(\thetawiggle,w\right)^{-1}\right)G\left(\thetaone,w\right)}_{2}\\
 & =\sup_{w\in W_{\delta}}\norm{H\left(\thetawiggle,w\right)^{-1}\left(H\left(\thetawiggle,w\right)-H\left(\thetaone,\onevec\right)\right)H\left(\thetaone,\onevec\right)^{-1}G\left(\thetaone,w\right)}_{2}\\
 & \le2\constop\sup_{w\in W_{\delta}}\norm{\left(H\left(\thetawiggle,w\right)-H\left(\thetaone,\onevec\right)\right)H\left(\thetaone,\onevec\right)^{-1}G\left(\thetaone,w\right)}_{2}\textrm{ (Lemma \ref{lem:continuous_invertibility})}\\
 & \le2\constop\sup_{w\in W_{\delta}}\norm{H\left(\thetawiggle,w\right)-H\left(\thetaone,\onevec\right)}_{op}\norm{H\left(\thetaone,\onevec\right)^{-1}G\left(\thetaone,w\right)}_{2}\\
 & \le2\constop\sup_{w\in W_{\delta}}\norm{H\left(\thetawiggle,w\right)-H\left(\thetaone,\onevec\right)}_{2}\norm{H\left(\thetaone,\onevec\right)^{-1}G\left(\thetaone,w\right)}_{2}\textrm{ (ordering of matrix norms)}\\
 & \le2\constop\constij\delta\sup_{w\in W_{\delta}}\norm{H\left(\thetaone,\onevec\right)^{-1}G\left(\thetaone,w\right)}_{2}\textrm{ (Lemma \ref{lem:gh_difference_from_one})}\\
 & \le2\constop[2]\constij\delta\sup_{w\in W_{\delta}}\norm{G\left(\thetaone,w\right)}_{2}\textrm{ (Assumption \ref{assu:paper_hessian})}\\
 & =2\constop[2]\constij\delta\sup_{w\in W_{\delta}}\norm{G\left(\thetaone,w\right)-G\left(\thetaone,\onevec\right)}_{2}\textrm{ (because }G\left(\thetaone,\onevec\right)=0\textrm{)}\\
 & \le2\constop[2]\constij\delta^{2}\textrm{ (Condition \ref{cond:paper_uniform_bound})}.
\end{align*}
\end{proof}
%






\subsection{Use cases\label{sec:use_cases}}

First, let us state a simple condition under which \coreassum
hold. It will help to have a lemma for the Lipschitz continuity.
\begin{lem}
Derivative Cauchy Schwartz. Let $a\left(\theta\right)=\left(a_{1}\left(\theta\right),...,a_{N}\left(\theta\right)\right)$
be an array of tensors with multi-index $i\in\left[D_{A}\right]$,
and let $\frac{\partial a\left(\theta\right)}{\partial\theta}=\left(\frac{\partial}{\partial\theta}a_{1}\left(\theta\right),...,\frac{\partial}{\partial\theta}a_{N}\left(\theta\right)\right)$
be an array of tensors of size $D\times D_{A}$. Then
\begin{align*}
\norm{\frac{\partial}{\partial\theta}\norm{a\left(\theta\right)}_{2}}_{2} & \le D_{A}\norm{\frac{\partial a}{\partial\theta}}_{2}.
\end{align*}
\end{lem}
\begin{proof}
By direct calculation,
\begin{align*}
\norm{\frac{\partial}{\partial\theta}\norm{a\left(\theta\right)}_{2}^{2}}_{2}^{2} & =\sum_{r=1}^{D}\left(\frac{\partial}{\partial\theta_{r}}\sum_{n=1}^{N}\sum_{i=1}^{D_{A}}a_{n,i}\left(\theta\right)^{2}\right)^{2}\\
 & =\sum_{r=1}^{D}\left(\sum_{n=1}^{N}\sum_{i=1}^{D_{A}}2a_{n,i}\left(\theta\right)\frac{\partial a_{n,i}\left(\theta\right)}{\partial\theta_{r}}\right)^{2}\\
 & \le\sum_{r=1}^{D}\left(2\sum_{i=1}^{D_{A}}\left(\sum_{n=1}^{N}a_{n,i}\left(\theta\right)^{2}\right)^{\frac{1}{2}}\left(\sum_{n=1}^{N}\left(\frac{\partial a_{n,i}\left(\theta\right)}{\partial\theta_{r}}\right)^{2}\right)^{\frac{1}{2}}\right)^{2}\\
 & \le\sum_{r=1}^{D}\left(2D_{A}^{2}\left(\frac{1}{D_{A}}\sum_{i=1}^{D_{A}}\sum_{n=1}^{N}a_{n,i}\left(\theta\right)^{2}\right)^{\frac{1}{2}}\left(\frac{1}{D_{A}}\sum_{n=1}^{N}\left(\frac{\partial a_{n,i}\left(\theta\right)}{\partial\theta_{r}}\right)^{2}\right)^{\frac{1}{2}}\right)^{2}\\
 & =4D_{A}^{2}\norm a_{2}^{2}\sum_{r=1}^{D}\norm{\frac{\partial a}{\partial\theta_{r}}}_{2}^{2}\\
 & =4D_{A}^{2}\norm a_{2}^{2}\norm{\frac{\partial a}{\partial\theta}}_{2}^{2}.
\end{align*}
By the chain rule,
\begin{align*}
\norm{\frac{\partial}{\partial\theta}\norm{a\left(\theta\right)}_{2}}_{2}^{2} & =\frac{1}{4\norm{a\left(\theta\right)}_{2}^{2}}\norm{\frac{\partial}{\partial\theta}\norm{a\left(\theta\right)}_{2}^{2}}_{2}^{2}\le D_{A}^{2}\norm{\frac{\partial a}{\partial\theta}}_{2}^{2}.
\end{align*}
\end{proof}
%
\begin{lem}
\label{lem:lipschitz_helper}Let $a\left(\theta\right)\in\mathbb{R}^{D\times D}$
be a continuously differentiable random matrix with a $D\times D\times D$
derivative tensor. (Note that the function, not $\theta$, is random.
For example, $\mbe\left[a\left(\theta\right)\right]$ is still a function
of $\theta$.) Suppose that $\mbe\left[\norm{a\left(\theta\right)}_{2}\right]$
is finite for all $\theta\in\Omega_{\theta}$. Then, for all $\theta_{1},\theta_{2}\in\Omega_{\theta}$,
\begin{align*}
\left|\mbe\left[\norm{a\left(\theta_{1}\right)}_{2}\right]-\mbe\left[\norm{a\left(\theta_{2}\right)}_{2}\right]\right| & \le\sqrt{\mbe\left[\sup_{\theta\in\Omega_{\theta}}\norm{\frac{\partial a\left(\theta\right)}{\partial\theta}}_{2}^{2}\right]}\norm{\theta_{1}-\theta_{2}}_{2}.
\end{align*}
\end{lem}
\begin{proof}
For any tensor $a$ with multi-index $i$,
\begin{align*}
\norm{\frac{\partial}{\partial\theta}\norm a_{2}^{2}}_{2}^{2} & =\sum_{r=1}^{D}\left(\frac{\partial}{\partial\theta_{r}}\norm a_{2}^{2}\right)^{2}\\
 & =\sum_{r=1}^{D}\left(\frac{\partial}{\partial\theta_{r}}\sum_{i=1}^{D_{A}}a_{i}^{2}\right)^{2}\\
 & =\sum_{r=1}^{D}\left(2\sum_{i=1}^{D_{A}}a_{i}\frac{\partial a_{i}}{\partial\theta_{r}}\right)^{2}\\
 & \le4\sum_{r=1}^{D}\sum_{i=1}^{D_{A}}a_{i}^{2}\sum_{i=1}^{D_{A}}\left(\frac{\partial a_{i}}{\partial\theta_{r}}\right)^{2}\textrm{ (Cauchy-Schwartz)}\\
 & =4\sum_{i=1}^{D_{A}}a_{i}^{2}\sum_{r=1}^{D}\sum_{i=1}^{D_{A}}\left(\frac{\partial a_{i}}{\partial\theta_{r}}\right)^{2}\\
 & =4\norm a_{2}^{2}\norm{\frac{\partial a}{\partial\theta}}_{2}^{2}.
\end{align*}

Consequently,
\begin{align*}
\norm{\frac{\partial}{\partial\theta}\norm{a\left(\theta\right)}_{2}}_{2}^{2} & =\norm{\frac{1}{2\norm{a\left(\theta\right)}_{2}}\frac{\partial}{\partial\theta}\norm{a\left(\theta\right)}_{2}^{2}}_{2}^{2}\\
 & =\frac{1}{4\norm{a\left(\theta\right)}_{2}^{2}}\norm{\frac{\partial}{\partial\theta}\norm{a\left(\theta\right)}_{2}^{2}}_{2}^{2}\\
 & \le\frac{4\norm{a\left(\theta\right)}_{2}^{2}}{4\norm{a\left(\theta\right)}_{2}^{2}}\norm{\frac{\partial}{\partial\theta}a\left(\theta\right)}_{2}^{2}\\
 & =\norm{\frac{\partial a\left(\theta\right)}{\partial\theta}}_{2}^{2}.
\end{align*}
So for any $\theta_{1},\theta_{2}\in\Omega_{\theta}$,
\begin{align*}
\left|\mbe\left[\norm{a\left(\theta_{1}\right)}_{2}\right]-\mbe\left[\norm{a\left(\theta_{2}\right)}_{2}\right]\right| & \le\mbe\left[\left|\norm{a\left(\theta_{1}\right)}_{2}-\norm{a\left(\theta_{2}\right)}_{2}\right|\right]\\
 & \le\mbe\left[\left(\sup_{\theta\in\Omega_{\theta}}\norm{\frac{\partial}{\partial\theta}\norm{a\left(\theta\right)}_{2}}_{2}\right)\right]\norm{\theta_{1}-\theta_{2}}_{2}\textrm{ (}\theta\textrm{ is not random)}\\
 & \le\mbe\left[\left(\sup_{\theta\in\Omega_{\theta}}\norm{\frac{\partial a\left(\theta\right)}{\partial\theta}}_{2}\right)\right]\norm{\theta_{1}-\theta_{2}}_{2}\\
 & \le\sqrt{\mbe\left[\sup_{\theta\in\Omega_{\theta}}\norm{\frac{\partial a\left(\theta\right)}{\partial\theta}}_{2}^{2}\right]}\norm{\theta_{1}-\theta_{2}}_{2}.
\end{align*}
The result follows. Note that the bound still holds (though vacuously)
if $\mbe\left[\sup_{\theta\in\Omega_{\theta}}\norm{\frac{\partial a\left(\theta\right)}{\partial\theta}}_{2}^{2}\right]$
is infinite.
\end{proof}
\begin{prop}
\label{prop:assumptions_hold}Let $\Omega_{\theta}$ be a compact
set. Let $g_{n}\left(\theta\right)$ be twice continuously differentiable
IID random functions. Define
\begin{align*}
h_{n}\left(\theta\right) & :=\frac{\partial g{}_{n}\left(\theta\right)}{\partial\theta}\\
r_{n}\left(\theta\right) & :=\frac{\partial^{2}g{}_{n}\left(\theta\right)}{\partial\theta\partial\theta},
\end{align*}
where $r_{n}\left(\theta\right)$ is a $D\times D\times D$ tensor.
Assume that

1a) $\mbe\left[\sup_{\theta\in\Omega_{\theta}}\norm{g_{n}\left(\theta\right)}_{2}^{2}\right]<\infty$;

1b) $\mbe\left[\sup_{\theta\in\Omega_{\theta}}\norm{h_{n}\left(\theta\right)}_{2}^{2}\right]<\infty$;

1c) $\mbe\left[\sup_{\theta\in\Omega_{\theta}}\norm{r_{n}\left(\theta\right)}_{2}^{2}\right]<\infty;$

2) $\mbe\left[h_{n}\left(\theta\right)\right]$ is non-singular for
all $\theta\in\Omega_{\theta}$;

3) We can exchange expectation and differentiation.

Then $\lim_{N\rightarrow\infty}P\left(\textrm{\coreassum\ hold}\right)=1.$
\end{prop}
%
\begin{proof}
    %
The proof follows from Theorems 9.1 and
9.2 of \citet{keener:2011:theoretical}. We will first show that the expected values of
the needed functions satisfy \coreassum, and then that the sample versions
converge uniformly.

By Jensen's inequality,
\begin{align*}
\mbe\left[\sup_{\theta\in\Omega_{\theta}}\norm{g_{n}\left(\theta\right)}_{2}\right] & =\mbe\left[\sqrt{\sup_{\theta\in\Omega_{\theta}}\norm{g_{n}\left(\theta\right)}_{2}^{2}}\right]\le\sqrt{\mbe\left[\sup_{\theta\in\Omega_{\theta}}\norm{g_{n}\left(\theta\right)}_{2}^{2}\right]}.
\end{align*}
Also, for the $i^{th}$ component of $g_{n}\left(\theta\right)$
\begin{align*}
\mbe\left[\sup_{\theta\in\Omega_{\theta}}\left|g_{n,i}\left(\theta\right)\right|\right] & \le\mbe\left[\sup_{\theta\in\Omega_{\theta}}\norm{g_{n}\left(\theta\right)}_{\infty}\right]\le\mbe\left[\sup_{\theta\in\Omega_{\theta}}\norm{g_{n}\left(\theta\right)}_{2}\right].
\end{align*}
By Theorem 9.1 of \citet{keener:2011:theoretical}, $\mbe\left[\norm{g_{n}\left(\theta\right)}_{2}^{2}\right]$
, $\mbe\left[\norm{g_{n}\left(\theta\right)}_{2}\right]$, and $\mbe\left[g_{n}\left(\theta\right)\right]$
are continuous functions of $\theta$, and because $\Omega_{\theta}$
is compact, they are each bounded. Similar reasoning applies to $h_{n}\left(\theta\right)$
and $r_{n}\left(\theta\right)$. Consequently we can define
\begin{align*}
\sup_{\theta\in\Omega_{\theta}}\mbe\left[\norm{g_{n}\left(\theta\right)}_{2}^{2}\right] & =:Q_{g}^{2}<\infty\\
\sup_{\theta\in\Omega_{\theta}}\mbe\left[\norm{h_{n}\left(\theta\right)}_{2}^{2}\right] & =:Q_{h}^{2}<\infty.
\end{align*}
Below, these constants will be used to satisfy \assuref{paper_smoothness}
and \assuref{paper_bounded} with high probability.

Because $\Omega_{\theta}$ is compact, $\mbe\left[h_{n}\left(\theta\right)\right]$
is continuous, $\mbe\left[h_{n}\left(\theta\right)\right]$ is non-singular,
and the operator norm is a continuous function of $\mbe\left[h_{n}\left(\theta\right)\right]$,
we can also define
\begin{align*}
\sup_{\theta\in\Omega_{\theta}}\norm{\mbe\left[h_{n}\left(\theta\right)\right]^{-1}}_{op} & =:Q_{op}<\infty.
\end{align*}
Below, this constant be used to satisfy \assuref{paper_hessian}
with high probability.

Finally, we turn to the Lipschitz condition. \lemref{lipschitz_helper}
implies that
\begin{align*}
\left|\mbe\left[\norm{h_{n}\left(\theta_{1}\right)}_{2}\right]-\mbe\left[\norm{h_{n}\left(\theta_{2}\right)}_{2}\right]\right| & \le\sqrt{\mbe\left[\sup_{\theta\in\Omega_{\theta}}\norm{r_{n}\left(\theta\right)}_{2}^{2}\right]}\norm{\theta_{1}-\theta_{2}}_{2}.
\end{align*}
Define
\begin{align*}
\Lambda_{h} & =\sqrt{\mbe\left[\sup_{\theta\in\Omega_{\theta}}\norm{r_{n}\left(\theta\right)}_{2}^{2}\right]},
\end{align*}
so that we have shown that $\mbe\left[\norm{h_{n}\left(\theta\right)}_{2}\right]$
is Lipschitz in $\Omega_{\theta}$ with constant $\Lambda_{h}$, which
is finite by assumption.

We have now shown, essentially, that the expected versions of the
quantities we wish to control satisfy \coreassum with $N=1$.
We now need to show that the sample versions satisfy \coreassum
with high probability, which will follow from the fact that the sample
versions converge uniformly to their expectations by
Theorem 9.2 of \citet{keener:2011:theoretical}.

First, observe that \assuref{paper_smoothness} holds with probability
one by assumption. For the remaining assumption choose an $\epsilon>0$,
and define
\begin{align*}
\constg & :=\sqrt{Q_{g}^{2}+\epsilon}\\
\consth & :=\sqrt{Q_{h}^{2}+\epsilon}\\
\constop & :=2Q_{op}\\
\liph & :=\sqrt{D^{4}\Lambda_{h}^{2}+\epsilon}.
\end{align*}

By \citet{keener:2011:theoretical} Theorem 9.2,
\begin{align*}
\sup_{\theta\in\Omega_{\theta}}\left|\frac{1}{N}\sum_{n=1}^{N}\norm{g_{n}\left(\theta\right)}_{2}^{2}-\mbe\left[\norm{g_{n}\left(\theta\right)}_{2}^{2}\right]\right| & \xrightarrow[N\rightarrow\infty]{p}0.
\end{align*}
Because
\begin{align*}
\sup_{\theta\in\Omega_{\theta}}\left|\frac{1}{N}\sum_{n=1}^{N}\norm{g_{n}\left(\theta\right)}_{2}^{2}\right| & >Q_{g}^{2}+\epsilon\ge\sup_{\theta\in\Omega_{\theta}}\mbe\left[\norm{g_{n}\left(\theta\right)}_{2}^{2}\right]+\epsilon\Rightarrow\\
\sup_{\theta\in\Omega_{\theta}}\left|\frac{1}{N}\sum_{n=1}^{N}\norm{g_{n}\left(\theta\right)}_{2}^{2}-\mbe\left[\norm{g_{n}\left(\theta\right)}_{2}^{2}\right]\right| & >\epsilon,
\end{align*}

we have
\begin{align*}
 & P\left(\sup_{\theta\in\Omega_{\theta}}\left|\frac{1}{N}\sum_{n=1}^{N}\norm{g_{n}\left(\theta\right)}_{2}^{2}\right|\ge Q_{g}^{2}+\epsilon\right)\le\\
 & \quad P\left(\sup_{\theta\in\Omega_{\theta}}\left|\frac{1}{N}\sum_{n=1}^{N}\norm{g_{n}\left(\theta\right)}_{2}^{2}-\mbe\left[\norm{g_{n}\left(\theta\right)}_{2}^{2}\right]\right|\le\epsilon\right),
\end{align*}
so
\begin{align*}
P\left(\sup_{\theta\in\Omega_{\theta}}\left|\frac{1}{N}\sum_{n=1}^{N}\norm{g_{n}\left(\theta\right)}_{2}^{2}\right|\ge\constg^{2}\right) & \xrightarrow[N\rightarrow\infty]{}0.
\end{align*}
An analogous argument holds for $\frac{1}{N}\norm{h_{n}\left(\theta\right)}_{2}^{2}$.
Consequently, $P\left(\textrm{Assumption \ref{assu:paper_bounded} holds}\right)\xrightarrow[N\rightarrow\infty]{}1$.

We now consider \assuref{paper_hessian}. Again, by \citet{keener:2011:theoretical} Theorem 9.2
applied to each element of the matrix $h_{n}\left(\theta\right)$,
using a union bound over each of the $D^{2}$ entries,
\begin{align*}
\sup_{\theta\in\Omega_{\theta}}\norm{\frac{1}{N}\sum_{n=1}^{N}h_{n}\left(\theta\right)-\mbe\left[h_{n}\left(\theta\right)\right]}_{1} & \xrightarrow[N\rightarrow\infty]{p}0.
\end{align*}
By the converse of \proprefref{operator_norm_continuity}, because
$\norm{\mbe\left[h_{n}\left(\theta\right)\right]^{-1}}_{op}\le Q_{op}$,
\begin{align*}
\norm{\left(\frac{1}{N}\sum_{n=1}^{N}h_{n}\left(\theta\right)\right)^{-1}}_{op} & >2Q_{op}=\constop\Rightarrow\\
\norm{\frac{1}{N}\sum_{n=1}^{N}h_{n}\left(\theta\right)-\mbe\left[h_{n}\left(\theta\right)\right]}_{1} & >\frac{1}{2}Q_{op}^{-1}.
\end{align*}
Consequently,
\begin{align*}
 & P\left(\norm{\left(\frac{1}{N}\sum_{n=1}^{N}h_{n}\left(\theta\right)\right)^{-1}}_{op}\ge\constop\right)\le\\
 & \quad P\left(\norm{\frac{1}{N}\sum_{n=1}^{N}h_{n}\left(\theta\right)-\mbe\left[h_{n}\left(\theta\right)\right]}_{1}\right)\xrightarrow[N\rightarrow\infty]{p}0,
\end{align*}
and $P\left(\textrm{Assumption \ref{assu:paper_hessian} holds}\right)\xrightarrow[N\rightarrow\infty]{}1.$

Finally, applying \lemref{lipschitz_helper} to $\frac{1}{\sqrt{N}}\norm{h\left(\theta_{2}\right)}_{2}$,
\begin{align*}
\left|\frac{1}{\sqrt{N}}\norm{h\left(\theta_{1}\right)}_{2}-\frac{1}{\sqrt{N}}\norm{h\left(\theta_{2}\right)}_{2}\right| & \le\sup_{\theta\in\Omega_{\theta}}\norm{\frac{\partial}{\partial\theta}\frac{1}{\sqrt{N}}\norm{h\left(\theta\right)}_{2}}_{2}\norm{\theta_{1}-\theta_{2}}_{2}\\
 & \le\frac{D^{2}}{\sqrt{N}}\sup_{\theta\in\Omega_{\theta}}\norm{r\left(\theta\right)}_{2}\norm{\theta_{1}-\theta_{2}}_{2}\\
 & =D^{2}\sqrt{\sup_{\theta\in\Omega_{\theta}}\frac{1}{N}\norm{r\left(\theta\right)}_{2}^{2}}\norm{\theta_{1}-\theta_{2}}_{2}.
\end{align*}
Consequently,
\begin{align*}
\left|\frac{1}{\sqrt{N}}\norm{h\left(\theta_{1}\right)}_{2}-\frac{1}{\sqrt{N}}\norm{h\left(\theta_{2}\right)}_{2}\right| & \ge\liph\norm{\theta_{1}-\theta_{2}}_{2}\Rightarrow\\
D^{2}\sqrt{\sup_{\theta\in\Omega_{\theta}}\frac{1}{N}\norm{r\left(\theta\right)}_{2}^{2}} & \ge\liph\Rightarrow\\
\sup_{\theta\in\Omega_{\theta}}\frac{1}{N}\norm{r\left(\theta\right)}_{2}^{2}-\sup_{\theta\in\Omega_{\theta}}\mbe\left[\norm{r_{n}\left(\theta\right)}_{2}^{2}\right] & \ge\frac{\liph^{2}}{D^{4}}-\sup_{\theta\in\Omega_{\theta}}\mbe\left[\norm{r_{n}\left(\theta\right)}_{2}^{2}\right]\Rightarrow\\
\sup_{\theta\in\Omega_{\theta}}\left|\frac{1}{N}\norm{r\left(\theta\right)}_{2}^{2}-\mbe\left[\norm{r_{n}\left(\theta\right)}_{2}^{2}\right]\right| & \ge\frac{\liph^{2}}{D^{4}}-\Lambda_{h}^{2}=\epsilon.
\end{align*}
However, again by \citet{keener:2011:theoretical} Theorem 9.2,
\begin{align*}
\sup_{\theta\in\Omega_{\theta}}\left|\frac{1}{N}\norm{r\left(\theta\right)}_{2}^{2}-\mbe\left[\norm{r_{n}\left(\theta\right)}_{2}^{2}\right]\right| & \xrightarrow[N\rightarrow\infty]{p}0,
\end{align*}
so $P\left(\textrm{Assumption \ref{assu:paper_lipschitz} holds}\right)\xrightarrow[N\rightarrow\infty]{}1$.
\end{proof}
